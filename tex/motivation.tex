
\section{Motivation}
Currently, the analyzer handles loops quite simply. More precisely, it unrolls 
them 4 times by default and then cuts the analysis of the path where the loop 
would have been unrolled more than 4 times. This behavior is reached by the 
above presented basic block visiting budget (restriction no. 4).

Loss in code coverage is one of the problems with this approach to loop
modeling. Specifically, in cases where the loop is statically known to make 
more than 4 steps, the analyzer do not analyze the code following the loop. 
Thus, the naive loop handling (described above) could lead to entirely 
unchecked code.
Listing \ref{UnrollMot} shows a small example for that.

\lstset{language=C,caption={Since the loop condition is known at every 
		iteration, the analyzer will not check the 'rest of the function' 
		part in the 
		current state.},label=UnrollMot}
\begin{lstlisting}
void foo() {
  int arr[6];
  for (int i = 0; i < 6; i++) {
    arr[i] = i;
  }
  /*rest of the function*/
}\end{lstlisting}

According to the budget rule concerning the basic block visit number, the
analysis of the loop stops in the fourth iteration even if the loop 
condition is simple enough to see that unrolling the whole loop would not be 
too much extra work relatively. Running out of the budget implies (in this 
case) that the rest of the function body remains unanalyzed, which may lead to 
not finding potential bugs.
Another problem can be seen on Listing \ref{WidenMot}:
\lstset{language=C++,caption={The loop condition is unknown but the 
analyzer 
		will not generate a simulation path where n $\ge$ 4 (which can 
		result coverage 
		loss).},label=WidenMot}
\begin{lstlisting}
int num();
void foo() {
  int n = 0;
  for (int i = 0; i < num(); ++i) {
    ++n;
  }
  /*rest of the function, n < 4 */
}\end{lstlisting}

This code fragment results in an analysis which keeps track of the values of 
\texttt{n} and \texttt{i} variables (this information is stored in the State). 
In every iteration of the loop the values are updated accordingly. Note that 
updating the
State means that a new node is inserted into the \texttt{ExplodedGraph} with the 
new 
values. Since the body of the \texttt{num()} function is unknown, the 
analyzer can not find out its return value. Thus it is considered as 
unknown. This circumstance
makes the graph to split into two branches. The first one belongs to the symbolic
execution of the loop body assuming that the loop condition is true. The 
other one simulates the case where the condition is false and the execution 
continues after the loop. This process is done for every loop iteration, 
however, at the 4th time, assuming the condition is true, the path will be cut 
short according to the budget rule.
Even though the analyzer generates paths to simulate the code after the loop 
in the above described case, the value of variable \texttt{n} will be 
always less than 4 on these paths and the rest of the function will only be 
checked assuming this constraint. Since the analysis did not had any 
precondition on the possible values of the variables, this constraint not 
necessary correct. Most of the time this method results in coverage loss as 
well, since the analyzer will ignore the paths where n is more than 4.
