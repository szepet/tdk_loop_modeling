\documentclass[oneside, a4paper, 12pt]{article}
\usepackage{thesis-style}
\usepackage{multirow}
\usepackage{amsmath,amsfonts,amssymb,amsthm,epsfig,epstopdf,titling,url,array}
\usepackage{mdframed}
\usepackage{tabularx, booktabs}
\usepackage{siunitx}
\usepackage{textcomp}

\sisetup{retain-explicit-plus}
\newcolumntype{L}[1]{>{\raggedright\arraybackslash}m{#1}}
\newcolumntype{C}[1]{>{\centering\arraybackslash}m{#1}}
\newcolumntype{Y}{>{\centering\arraybackslash}X}
\newcommand\TB{\rule[-1.5ex]{0ex}{4.7ex}} % "top&bottom" strut

\usepackage{listings}
\usepackage{color}

\usepackage{tcolorbox}
\usepackage[hidelinks]{hyperref}
\usepackage[inline]{enumitem}

\newcolumntype{Y}{>{\centering\arraybackslash}X}

\tcbuselibrary{theorems}

\theoremstyle{definition}
\newtheorem{definition}{Definíció}[section]
\newtheorem{assumption}{Feltétel}[section]
\newtheorem{stmt}{Állítás}[section]
\newtheorem{prof}{Bizonyítás}[section]

\newtcbtheorem[number within=section]{algorithm}{Algoritmus}%
{colback=green!5,colframe=green!35!black,fonttitle=\bfseries}{th}

\definecolor{mygreen}{rgb}{0,0.6,0}
\definecolor{mygray}{rgb}{0.5,0.5,0.5}
\definecolor{mymauve}{rgb}{0.58,0,0.82}
\definecolor{mygrayback}{rgb}{0.9,0.9,0.9}
\lstset{ %
backgroundcolor=\color{mygrayback},   % choose the background color; you 
%must
%add \usepackage{color} or \usepackage{xcolor}; should come as last argument
basicstyle=\small,        % the size of the fonts that are used for the code
breakatwhitespace=false,
breaklines=true, 
captionpos=b,
commentstyle=\color{mygreen},    % comment style
extendedchars=true,              % lets you use 
%frame=single,	                   % adds a frame 
keepspaces=true,                 % keeps spaces in 
keywordstyle=\color{blue},       % keyword style
language=C++,                 %
numbers=left,                    % where to put the line-numbers; possible
%values are (none, left, right)
numbersep=5pt,                   % how far the line-numbers are from the 
%code
numberstyle=\footnotesize\color{mygray}, % the style that is used for the
%line-numbers
rulecolor=\color{black},         % if not set, the frame-color may be 
%changed
%on line-breaks within not-black text (e.g. comments (green here))
showspaces=false,                % show spaces everywhere adding particular
%underscores; it overrides 'showstringspaces'
showstringspaces=false,          % underline spaces within strings only
showtabs=false,                  % show tabs within strings adding 
%particular
%underscores
stepnumber=1,                    % the step between two line-numbers. If 
%it's
%1, each line will be numbered
stringstyle=\color{mymauve},     % string literal style
tabsize=2,	                   % sets default tabsize to 2 spaces
title=\lstname                   % show the filename of files included with
%\lstinputlisting; also try caption instead of title
}



% Töltsd ki a saját szakdolgozatod adataival
\def\CIM{
Improved symbolic execution loop modeling for C like languages}
\def\SZERZO{Szécsi Péter}
\def\VEDESEVE{2018}

\def\TANSZEK{Programozási Nyelvek és Fordítóprogramok Tanszék}
\def\TEMAVEZETTO{Horváth Gábor}
\def\TEMAVEZETTOBEOSZTAS{Doktorandusz}
\def\TEMAVEZETO{Porkoláb Zoltán}
\def\TEMAVEZETOBEOSZTAS{Egyetemi docens}

\title{\CIM}
\author{\SZERZO}
\date{\VEDESEVE}	

\begin{document}
\pagestyle{empty}

% belső fedőlap
\begin{titlepage}

\begin{minipage}{0.40\linewidth}
\includegraphics[scale=0.3]{img/elte-cimer}
\end{minipage}
\begin{minipage}{0.50\linewidth}
\begin{center}
Eötvös Loránd University \\
Faculty of Informatics \\
\TANSZEK
\end{center}
\end{minipage}

\hrule
\vfill

\begin{center}
\Huge
\textbf{\CIM}
\normalsize
\end{center}

\vfill

\begin{minipage}[t]{0.5\linewidth}
\noindent
Supervisors:\\
\textbf{\TEMAVEZETO} \hfill \textbf{\TEMAVEZETTO}\\
\TEMAVEZETOBEOSZTAS  \hfill \TEMAVEZETTOBEOSZTAS
\end{minipage}
\begin{minipage}[t]{0.5\linewidth}
\begin{flushright}
Author: \\
\textbf{\SZERZO} \\
Computer Science MSc
\end{flushright}
\end{minipage}

\vfill

\begin{center}
\includegraphics[scale=0.1]{img/unkp}\\
\textit{\small{This study was supported by the \'UNKP-17-2 New National 
Excellence Program of the Hungarian Ministry of Human Capacities.}}


Budapest, \VEDESEVE
\end{center}

\end{titlepage}

\cleardoublepage

% tartalomjegyzék
\tableofcontents
\clearpage

\pagestyle{plain}
\setcounter{page}{4}

% tartalom
\begin{abstract}
The LLVM Clang Static Analyzer is a source code analysis tool which aims to 
find bugs in C,
C++, and Objective-C programs using symbolic execution, i.e. it simulates the
possible execution paths of the code. Currently the simulation of the loops is
somewhat naive (but efficient), unrolling the loops a predefined constant 
number of times. However, this approach can result in a loss of coverage in 
various cases.

This study aims to introduce two alternative approaches which can 
extend the current method and can be applied simultaneously: 
\begin{enumerate*} [label={(\arabic*)}, noitemsep]
	\item determining loops worth to fully unroll with applied heuristics, and
	\item using a widening mechanism to simulate an arbitrary number of 
	iteration steps.
\end{enumerate*}
These methods were evaluated on numerous open source projects, and 
proved to increase coverage in most of the cases. This work also laid the
infrastructure for future loop modeling improvements.
\end{abstract}

% 2012 ACM Computing Classification System (CSS) concepts
% Generate at 'http://dl.acm.org/ccs/ccs.cfm'.

% \ccsdesc[500]{Software and its engineering~General programming languages}
% \ccsdesc[300]{Theory of computation~Program analysis}
% end generated code

% Legacy 1998 ACM Computing Classification System categories are also
% supported, but not recommended.
%\category{CR-number}{subcategory}{third-level}[fourth-level]
%\category{D.3.0}{Programming Languages}{General}
%\category{F.3.2}{Logics and Meanings of Programs}{Semantics of Programming
%Languages}[Program analysis]

\section{Introduction}
During software development it is natural to make mistakes. Consequently,
writing various test cases is required in order to validate the behavior
of the program. In addition to the costs of test writing, it is possible that the developers fail
to cover all possible critical cases. Furthermore, test writing and running
often happens later than code development, but the costs of error correction
increases proportionally to elapsed time  \cite{fixcost}. This proves that 
testing alone is not necessarily sufficient to ensure code quality.  %Moreover,
%there are specific properties of a software - e.g. coding convention following 
%- which can not be checked by test cases.

The static analysis tools offer a different approach for code validation 
\cite{Zhivich2009} \cite{Bessey2010}, moreover, they can potentially check for 
some characteristics of the code -- which cannot be verified by testing -- e.g. 
the adherence to conventions.

TODO:
Ezen programok a forráskódot tekintik bemenetnek, amelyből egy modellt
építenek fel és ezt felhasználva vonnak le következtetéseket. A felépített
modell komplexitásától függően az elemzés hatékonysága, precizitása és
időigénye is változik. A következtetések eredményeként az adott program
\textit{hibáinak} egy listáját kapjuk. Ez azonban nem feltétlenül a program 
hibás működését jelenti,
hanem egyéb javítási lehetőségekre is utalhat. Esetenként emberek számára
kevéssé olvasható vagy teljesítménybeli problémát okozó kódrészleteket
találhatunk. Ezeken túl törékeny vagy potenciális hordozhatósági problémákkal
rendelkező kódrészleteket is hibásként értelmezzük, tehát a kimenet
változatos formát ölthet.

Érdemes megjegyezni, hogy a statikus elemzést hibakeresésen túl más célokra is
szokták alkalmazni, mint például optimalizációs lehetőségek keresése. A 
dolgozatomban azonban elsősorban a statikus elemzésre mint a hibakeresés egy
eszközére tekintek.
 
 Unfortunately, it is impossible to detect every 
bug only by using static analysis \cite{Rice:53}. Static analyzer tools might 
not be able to discover some bugs (these are called false negatives) or report 
correct code snippets as incorrect (false positives, see Fig \ref{fig:bes}). In 
the industry the goal of these tools is to keep the ratio of the false positive 
reports low while still be able to find real bugs.

\begin{figure}[!h]
	\begin{center}
		\begin{tabular}{ | l | c | r | }
			\hline
			& True error   & Non-error \\ \hline
			Reported error   & True Positive  & False Positive   \\ \hline
			No error report & False Negative & True Negative    \\
			\hline
		\end{tabular}
	\end{center}
	\caption{The different types of the static analysis results.}
	\label{fig:bes}
\end{figure}

Its important to note that another advantage of the static analyzer tools 
(against test cases) that the analysis happens without executing the source 
code, so it can be used without the running environment. Moreover, the code 
coverage of the analysis does not depends on the already written test case set. 


The LLVM Clang Static Analyzer is a source code analysis tool which aims to 
find bugs in C,
C++, and Objective-C programs using - one of the most precise analysis methods 
- symbolic execution  \cite{King1975} \cite{Hampapuram2005} , i.e. it simulates 
the possible execution paths of the code. During symbolic execution, the 
program is being interpreted, on a function-by-function basis, without any 
knowledge about the runtime environment. It builds up and traverses an inner 
model of the execution paths, called \texttt{ExplodedGraph}, for each analyzed 
function. (TODO: Ez a módszer
az elágazások számában exponenciális elemzési időt és modell méretet 
eredményez. )

\section{The Clang Static Analyzer}

The Static Analyzer -- as it is indicated by its name -- build around the Clang 
compiler \cite{Lattner2008}. An important technical note is that the building 
of the \texttt{ExplodedGraph} is based on the \texttt{Control Flow Graph} 
(\texttt{CFG}) of the functions. The \texttt{CFG} represents a source-level, 
intra-procedural control flow of a 
statement. This statement can potentially be an entire function body, or just a 
single expression. The \texttt{CFG} consists of \texttt{CFGBlocks} which are 
simply containers of statements. The \texttt{CFGBlocks}s essentially represent 
the basic blocks of the code but can contain some extra custom information. 
Although basic blocks and \texttt{CFGBlocks} are technically different, in the 
rest of the article the term basic blocks will be used for \texttt{CFGBlocks} 
as well for the sake of easier understanding and better illustration.

Thus during the analysis -- based on the function CFGs -- an 
\texttt{ExplodedGraph} is built up. A 
node of this graph (called \texttt{ExplodedNode}) contains a 
\texttt{ProgramPoint} (which determines the location) and a \texttt{State} 
(which contains any known information at that point). Its paths from the root 
to the leaves are modeling the different execution paths of the analyzed 
function. Whenever the execution encounters a branch, a corresponding branch 
will be created in the \texttt{ExplodedGraph} during the simulated 
interpretation.
Hence, branches lead to an exponential number of \texttt{ExplodedNode}s.
This combinatorial explosion is handled in the Static Analyzer by stopping 
the analysis when given conditions are fulfilled. Terminating the analysis 
process may cause loss of potential true positive results, but it is 
indispensable for maintaining a reasonable resource consumption regarding the 
memory and CPU usage. 

These conditions are modeled by the concept of budget.
The budget is a collection of limitations on the shape of the \texttt{ExplodedGraph}.
These limitations include:
\begin{enumerate}  
	\item The maximum number of traversed nodes in the \texttt{ExplodedGraph}. 
	If 	this number is reached then the analysis of the simulated function stops.
	\item The size of the simulated call stack. When a function call is 
	reached then the analysis continues in its body as if it was inlined to the 
	place of call (interprocedural). There are several heuristics that may control 
	the	behavior of inlining process. For example the too large functions are 
	not	inlined at all, and the really short functions are not counted in the 
	size of	call stack.
	\item The number of times a function is inlined. The idea behind this
	constraint is that the more a function is analyzed, the less likely it 
	is that a 	bug will appear in it. If this number is reached then that 
	function will not be inlined again in this \texttt{ExplodedGraph}.
	\item The number of times a basic block is processed during the 
	analysis. This	constraint limits the number of loop iterations. When this 
	threshold is reached the currently analyzed execution path is aborted.
	The budget expression can be used in two ways. Sometimes it means the
	collection of the limitations above, sometimes it refers to one of these
	limitations. This will always be distinguishable from the context.
\end{enumerate}


\section{Motivation}
Currently, the analyzer handles loops quite simply. More precisely, it unrolls 
them 4 times by default and then cuts the analysis of the path where the loop 
would have been unrolled more than 4 times. This behavior is reached by the 
above presented basic block visiting budget.

Loss in code coverage is one of the problems with this approach to loop
modeling. Specifically, in cases where the loop is statically known to make 
more than 4 steps, the analyzer do not analyze the code following the loop. 
Thus, the naive loop handling (described above) could lead to entirely 
unchecked code.
Listing \ref{UnrollMot} shows a small example for that.

\lstset{language=C,caption={Since the loop condition is known at every 
		iteration, the analyzer will not check the 'rest of the function' 
		part in the 
		current state.},label=UnrollMot}
\begin{lstlisting}
void foo() {
  int arr[6];
  for (int i = 0; i < 6; i++) {
    arr[i] = i;
  }
  /*rest of the function*/
}\end{lstlisting}

According to the budget rule concerning the basic block visit number, the
analysis of the loop stops in the fourth iteration even if the loop 
condition is simple enough to see that unrolling the whole loop would not be 
too much extra work relatively. Running out of the budget implies (in this 
case) that the rest of the function body remains unanalyzed, which may lead to 
not finding potential bugs.
Another problem can be seen on Listing \ref{WidenMot}:
\lstset{language=C++,caption={The loop condition is unknown but the 
analyzer 
		will not generate a simulation path where n $\ge$ 4 (which can 
		result coverage 
		loss).},label=WidenMot}
\begin{lstlisting}
int num();
void foo() {
  int n = 0;
  for (int i = 0; i < num(); ++i) {
    ++n;
  }
  /*rest of the function, n < 4 */
}\end{lstlisting}

This code fragment results in an analysis which keeps track of the values of 
\texttt{n} and \texttt{i} variables (this information is stored in the State). 
In every iteration of the loop the values are updated accordingly. Note that 
updating the
State means that a new node is inserted into the \texttt{ExplodedGraph} with the 
new 
values. Since the body of the \texttt{num()} function is unknown, the 
analyzer can not find out its return value. Thus it is considered as 
unknown. This circumstance
makes the graph to split into two branches. The first one belongs to the symbolic
execution of the loop body assuming that the loop condition is true. The 
other one simulates the case where the condition is false and the execution 
continues after the loop. This process is done for every loop iteration, 
however, at the 4th time, assuming the condition is true, the path will be cut 
short according to the budget rule.
Even though the analyzer generates paths to simulate the code after the loop 
in the above described case, the value of variable \texttt{n} will be 
always less than 4 on these paths and the rest of the function will only be 
checked assuming this constraint. This can result in coverage loss as well, 
since the analyzer will ignore the paths where n is more than 4.

\section{Proposed Solution}
%The current loop handling method in the Clang Static Analyzer is too 
%strict.
In this section two solutions are presented to resolve the above mentioned
limitations on symbolic execution of loops in the Clang Static Analyzer. It 
is important to note that these enhancements are incremental in the sense 
that the original method is brought back on examples which are too complex to handle at the moment. For the sake of simplicity a 
"division by zero" will illustrate the bug we intend to find in the following examples.

\subsection{Loop Unrolling Heuristics}
Loop unrolling means we have identified heuristics and patterns (such as 
loops
with small number of branches and small known static bound) in order to find
specific loops which are worth to be completely unrolled. This idea is 
inspired by the following example:

\lstset{language=C++,caption={Complete unrolling of the loop makes it possible 
to find the division by zero error.},label=UnrollEx}
\begin{lstlisting}
void foo() {
  for (int i = 0; i < 6; i++) {
    /*simple loop which does not
    change 'i' or split the state*/
  }
  int k = 0;
  int l = 2/k; // Division by zero
}\end{lstlisting}

In the current solution a loop has to fulfill the following conditions in order 
to be unrolled:
\begin{enumerate}  
	\item The loop condition should arithmetically compare a variable -- 
	which is known at the beginning of the loop -- to a literal (like: 				\texttt{i~<~6} or \texttt{6~$\ge$~i})
	\item The loop should change the loop variable only once in its body 
	and the	difference needs to be constant. (This way the maximum number of 
	steps can be estimated.)
    \item There is no alias created to the loop variable.
	\item The estimated number of steps should be less than 128. (Simulating 
	loops which takes thousands of steps because they could single handedly 
	exhaust the budget.)
	\item The loop must not generate new branches or use \texttt{goto} 
	statements.
\end{enumerate}

By using this method, the bug on the Listing \ref{UnrollEx} example can be 
found successfully.

\subsection{Loop Widening}
The final aim of widening is quite the same as the unrolling, to increase 
the coverage of the analysis. However, it reaches it in a very different way. 
During widening the analyzer simulates the execution of an arbitrary number 
of iterations. The analyzer already has a widening algorithm which reaches 
this behavior by discarding all of the known information before the last step 
of the loop. So the analyzer creates the paths for the first 3 steps and 
simulate them as usual, but in order to avoid losing the first precise simulation branches, the widening (means the invalidating) happens 
before the 4th step. 
This way the coverage will be increased, however, this method is disabled by 
default, since it can easily result in too much false positives. Consider the 
example on Listing \ref{WidenProb}.
\\

\lstset{language=C++,caption={Invalidating every known information (even 
those which are not modified by the loop) can easily result in false 	
positives.},label=WidenProb}
\begin{lstlisting}
int num();
void foo() {
  bool b = true;
  for (int i = 0; i < num(); ++i) {
    /*does not changes 'b'*/
  }
  int n = 0;
  if (b)
    n++;
  n = 1/n; // False positive:
           // Division by zero
}\end{lstlisting}
In this case the analyzer will create and check that unfeasible path where 
the variable \texttt{b} is false, so \texttt{n} is not incremented and lead 
into a division by zero error. Since this execution path would never be 
performed while running the analyzed program, it is considered a false positive. My aim 
was to give a more precise approach for widening. There was already conversation within the community about some possible enhancements \cite{Widening}.

One of the main principles is that the analysis should still continue after the block 
visiting budget is exhausted and the information of only those variables should be invalidated which are possibly modified by the loop, e.g. a statement, like \texttt{arr[i] = i} where \texttt{i} is the loop variable, means that we discard the data on the whole \texttt{arr} array but nothing else).
For this I developed a solution which checks every possible way in which a 
variable can be modified in the loop. Then these cases are evaluated and if it 
encounters a modified variable which cannot be handled by the invalidation 
process (e.g.: a pointer variable), then the loop will not be widened and we
return to the conservative method. This mechanism ensures that we do not create 
nodes that contain invalid states.
This approach helps us to cover cases and find bugs like the one illustrated on 
Listing \ref{WidenEx} without reporting false positives  
presented on Listing \ref{WidenProb}.

\lstset{language=C++,caption={Invalidating the information on only the possible 
changed variables can result higher coverage (while limiting the number of the 
found false positives).}, label=WidenEx}
\begin{lstlisting}
int num();
void foo() {
  int n = 0;
  for (int i = 0; i < num(); ++i) {
    ++n;
  }
  if (n > 4) {
    int k = 0;
    k = 1/k;  // Division by zero error
  }
}\end{lstlisting}

The bug is found by invalidating the known information on variable
\texttt{n} (and \texttt{i} as well). This makes the analyzer to create a 
branch
where it checks the body of the \texttt{if} statement and finds the bug.
However, this solution has its own limitations when dealing with nested 
loops. 
Consider the case on Listing \ref{WidenEx2}.
\lstset{language=C++,caption={The naive widening method does not handle 
well the nested loops. In this example the outer loop will not be widened.},label=WidenEx2}
\begin{lstlisting}
int num();
void foo() {
  int n = 0;
  for (int i = 0; i < num(); ++i) {
    ++n;
    for (int j = 0; j < 4; ++j) {
      /*body that does not change n*/
    }
  }
  /*rest of the function, n <= 1 */
}\end{lstlisting}

In this scenario, when the analyzer first step into the outer loop (so it assumes
that \texttt{i < num()} is true) and encounter the inner loop, it consumes
its (own) block visiting budget. (This implies that it will be widened, although
in this case it means that only the inner loop counter (\texttt{j}) information
is discarded.) After moving on to the next iteration, we may assume that we
are on the path where the outer loop condition is true again. Due to the fact that the budget was already exhausted in the previous iteration, the next visit of the first
basic block of the inner loop (the condition) means that this path will be
completely cut off and not analyzed. This results in the outer loop not reaching the step number where it would been widened. Furthermore, the outer loop
will not even reach the 3rd step, even the 2nd is stopped at in its body
(as described above). This causes the problem that even though the 
loop widening method is used, the rest of the function will be analyzed by the 
assumption \texttt{n <= 1}.

In order to deal with the above described nested loop problem, I have 
implemented a replay mechanism. This means that whenever we encounter an inner 
loop which already consumed its budget, we replay the analysis process of the 
current step of the outer loop after performing a widening first. This ensures
the creation of a path which assumes that the condition is false and simulates 
the execution after the loop while the possibly changed information are 
discarded. This way the analyzer will not exclude some feasible path 
because of the simple loop handling which solves the problem.

An additional note to the widening process is that it makes sense to analyze the branch 
where the condition is true with the widened State as well. The example on 
Listing \ref{WidenNested} shows a case where this is useful.
\\
\\
\lstset{language=C++,caption={The replay mechanism successfully helps us to 
		find the possible error the outer loop.},label=WidenNested}
\begin{lstlisting}
int num();
void foo() {
  int n = 0;
  int i;
  for (i = 0; i < num(); ++i) {
    if (i == 7) {
      break;
    }
    for (int j = 0; j < 4; ++j) {/* */}
  }
  int n = 1 / (7 - k);
         // ^ Possible division by zero
}\end{lstlisting}
This way the analyzer will produce a path where the value of \texttt{i} is 
known
to be 7, so it will be able find the possible division by zero error.
	
\subsection{Infrastructure improvements}\label{sec:inf}
The discussed approaches heavily rely on the fact that we are able to 
perform the following actions:
\begin{enumerate}
	\item Decide on any \texttt{ExplodedNode} whether it simulates a body of a 
	loop or not,
	\item Recognize the entering and exiting point of a loop on the simulated 
	path.
\end{enumerate}

However, this information was not provided by the analyzer earlier. Considering the 
lexical nature of the loops, their entrance and exit points can unambiguously 
fit into the \texttt{CFG}. 

The \texttt{ProgramPoint} provides the \texttt{LocationContext} which implements a stack data structure for having the information on the different locations of the code.  This implies that the callstack can be represented via this structure in a straightforward manner (and is contained by the \texttt{LocationContext}).
Since storing the currently simulated loops fits into a stack data structure as well, this information -- called \texttt{LoopContext} -- understandably was implemented as the part of the \texttt{LocationContext} too instead of having them in the \texttt{Store}. Both the \texttt{Store} and the \texttt{LocationContext} are part of an \texttt{ExplodedNode} in form of a pointer. However, these structures use copy-on-write semantics, and the \texttt{LocationContext} changes way less times, this decision saves us memory. 

\section{Evaluation}

The effect of the described loop modeling approaches was measured on various 
C/C++ open source projects. These are listed on Table \ref{tab:lines}.

\begin{table}[!htb]
	\centering
\begin{tabular}{ |c||c|c| }
	\hline
	Project & LoC & Language \\
	\hline
	TinyXML & 20k & C++ \\
	\hline
	Curl & 21k & C  \\ 
	\hline
	Redis & 40k & C \\ 
	\hline
	Xerces & 228k & C++ \\ 
	\hline
	Vim & 540k & C \\ 
	\hline
	OpenSSL & 550k & C  \\ 
	\hline
	PostgreSQL & 950k & C \\ 
	\hline
	FFmpeg & 1080k & C \\	
	\hline
\end{tabular}
\caption{The projects used for profiling, their length in code lines, and 
language.}\label{tab:lines}
\end{table}

\subsection{Coverage and the number of explored paths}
Keeping track of these statisics are already part of the analyzer. The 
coverage percentage is based on the ratio of the visited and the total number 
of basic blocks in the analyzed functions (instead of the number of visited statements),
which results in a small imprecision. 
It is important to note that the introduced loop modeling methods require having 
additional loop entrance and exit point information (described in section 
\ref{sec:inf}) in the \texttt{CFG}. This can lead to having more 
basic blocks in the \texttt{CFG} and it can affect the statistics. As a result, 
even statistics produced by using the current loop modeling 
approach were measured with this information added to the \texttt{CFG}.

The coverage and the number of explored paths are generated for every 
translation unit and then summarized. This means that header files which 
are included in more than one translation unit can influence more statistics.
However, by using this summarization process consistently for every measurement 
the results reflect the reality. 

The tables presented in this section summarize measurement results using 
different loop modeling approaches: the current practice (denoted by Normal)
and the hereby introduced loop unrolling (Unroll) and loop widening (Widen)
methods separately and simultaneously (U+W).

\begin{table}[!htb]
	\centering
\begin{tabular}{ |c||c|c|c|c| } 
	\hline
	Project & Normal & Unroll & Widen & U + W \\
	\hline \hline
	TinyXML & 84.2 & 84.2 & 85.1 & 85.1 \\ 
	\hline
	Curl & 76.2 & 76.9 & 77.7 & 77.2 \\ 
	\hline
	Redis & 68.5 & 69.1 & 68.5 & 71.3 \\ 
	\hline
	Xerces & 92.3 & 92.4 & 92.7 & 92.7 \\ 
	\hline
	Vim & 60.4 & 60.6 &	60.6 & 60.7 \\ 
	\hline
	OpenSSL & 97.4 & 97.5 & 97.7 & 97.7 \\ 
	\hline
	PostgreSQL & 76.9 &	77.0 & 76.9 & 76.9 \\ 
	\hline
	FFmpeg & 86.1 & 86.3 & 87.0 & 86.8 \\	
	\hline
\end{tabular}
\caption{The code coverage of the analysis on the evaluated projects expressed 
in percentage.}\label{tab:cov}
\end{table}

Table \ref{tab:cov} shows the coverage difference using the introduced 
approaches. On most of the projects, analysis coverage was strictly increased by
using any of the proposed approaches. The widening method had a stronger 
influence on the coverage in the average case. However, the complete unroll of 
specific loops could result in a higher coverage as well (e.g. Curl, Redis).
In general, enabling both of them was the most beneficial with respect to the 
coverage.

\begin{table}[!htb]
	\centering
\begin{tabular}{ |c||r|r|r|r| } 
	\hline
	Project & Normal & Unroll & Widen & U + W \\
	\hline \hline
	TinyXML & 14\,452 & 15\,460 & 14\,765 & 15\,773 \\ 
	\hline
	Curl & 18\,272 & 18\,577 & 28\,835 & 24\,279 \\
	\hline
	Redis & 69\,857 & 70\,097 & 98\,446 & 100\,929 \\ 
	\hline
	Xerces & 395\,615 & 398\,077 & 430\,989 & 433\,358 \\ 
	\hline
	Vim & 155\,451 & 157\,266 & 188\,136 & 173\,121 \\ 
	\hline
	OpenSSL & 687\,175 & 687\,932 & 700\,464 & 701\,013 \\ 
	\hline
	PostgreSQL & 382\,660 & 383\,874 & 453\,188 & 419\,118  \\ 
	\hline
	FFmpeg & 466\,613 & 458\,480 & 571\,399 & 521\,725  \\ 		
	\hline
\end{tabular}
\caption{The numbers of explored execution paths using different loop modeling
approaches.}\label{tab:pathnum}
\end{table}
Table \ref{tab:pathnum} presents the numbers of analyzed execution paths.
As expected, both introduced loop modeling methods resulted in a higher number of
simulated paths on (almost) all of the projects. The only exception is the unrolling 
approach on the FFmpeg project, which caused the budget limiting the number of 
traversed \texttt{ExplodedNode}s to exhaust earlier, slightly decreasing 
the number of checked paths. Enabling both of the features resulted in similar or 
fewer number of explored paths than the runs using only widening. 
This effect can be explained in two ways:   
\begin{enumerate*} [label={(\arabic*)}, noitemsep]
	\item the analyzer prefers to completely unroll loops rather than widen them, 
    which results in a more precise modeling of the state and can exclude unfeasible 
    paths,
	\item the simultaneous use of the methods can lead to exhausting the budget on earlier paths, where the analysis will be terminated.
\end{enumerate*}

\subsection{Found bugs}
\begin{table}[!htb]
	\centering
\begin{tabular}{ |c||c|c|c|c| } 
	\hline
	Project & Normal & Unroll & Widen & U + W \\
	\hline \hline
	TinyXML & 1 & 1 & 3 & 3 (+200\%) \\
	\hline
	Curl & 16 & 16 & 16 & 16 (0\%) \\ 
	\hline
	Redis & 55 & 58 & 55 & 59 (+7.27\%) \\ 
	\hline
	Xerces & 62 & 62 & 61 & 61 (-1.61\%) \\ 
	\hline
	Vim & 74 & 74 & 76 & 78 (+5.4\%) \\
	\hline
	OpenSSL & 152 & 152 & 153 & 153 (+0.66\%) \\ 
	\hline
	PostgreSQL & 323 & 323 & 327 & 331 (+2.48\%) \\ 
	\hline
	FFmpeg & 425 & 420 & 423 & 454 (+6.82\%) \\ 		
	\hline
\end{tabular}
\caption{The number of bug reports produced by the analyzer.} \label{tab:reportnum}
\end{table}
The number of bug reports using the different loop modeling methods can be seen
in Table \ref{tab:reportnum}. The increase in analysis coverage and in the number of 
checked paths usually implies an increased number of found bugs, which indeed can be 
observed on the numbers. However, it is important to note that the upsurge of the number 
of explored execution paths described in Table \ref{tab:pathnum} considerably outweighs 
the moderate rise in the number of bug reports.
Since the loop widening method creates more new paths by discarding informations on the values of variables, it could introduce the risk of analyzing paths that lead to false positives.
However, from the results it seems that this was not a problem in practice: relative to the increase in the number of analyzed paths, the number of reports hardly increased. Moreover, based on studying the environment of the found bugs, the ratio of false positive findings was low (beside some clear true positive) among the newly detected bugs.

\subsection{Analysis time}

\begin{table}[!htb]
	\centering
\begin{tabular}{ |c||c|c|c|c| } 
	\hline
	Project & Normal & Unroll & Widen & U + W \\
	\hline \hline
	TinyXML & 0:51 & 0:51 & 0:52 & 0:52 (+2\%) \\ 
	\hline
	Curl & 0:50 & 1:06 & 0:55 & 1:05 (+30\%) \\ 
	\hline
	Redis & 2:06 & 2:11 & 2:28 & 2:10 (+3\%) \\ 
	\hline
	Xerces & 3:38 & 3:34 & 3:37 & 3:39 (+0.5\%) \\ 
	\hline
	Vim & 3:11 & 3:26 & 3:18 & 3:27 (+3\%) \\
	\hline
	OpenSSL & 2:04 & 2:22 & 2:13 & 2:19 (+8.3\%) \\
	\hline
	PostgreSQL & 7:03 & 8:32 & 7:48 & 7:59 (+13\%) \\
	\hline
	FFmpeg & 9:40 & 10:22 & 10:14 & 11:20 (+17\%) \\
	\hline
\end{tabular}
\caption{Average measured time of the analysis expressed in minutes. 
(Average of 5 runs.)}\label{tab:time}
\end{table} 
The running time on different projects is showed in Table \ref{tab:time}.
Although the widening method lead into more analyzed execution paths, the analysis time increase was more intense after enabling the unrolling process. This is possible due to the fact that unrolling leads to long paths where the \texttt{State} usually contains more information (constraints on variable values), which is very expensive in respect of running time. 
In general there was only a minor increase in the analysis time at all examined projects which suggests a good scalability of the proposed improvements.

\section{Conclusion}

Two alternative approaches was introduced for improving the simulation of loops during symbolic 
execution. These were implemented and subsequently evaluated on various open 
source projects, with a clear improvement of code coverage in general. The new methods make it possible to explore previously skipped, feasible execution paths, especially when both of them are used in conjunction.

The required changes done to the underlying infrastructure should also ease the 
implementation of future enhancements. In particular, information tracked by the 
analysis about location contexts were expanded with additional fields.
While code coverage was measured to have increased by an average of 0.8\% and the 
number of explored execution paths by an average of 16\%, there 
was a noticeable performance penalty as well. A general increase in the runtime was observed, with an average of 9.5\% . The number of simulated paths also increased proportionally with the time taken, suggesting this time was well 
spent. In conclusion, if the user does not mind taking $\sim$10\% more time
for a more comprehensive analysis, then it is beneficial to enable the proposed
feature set by default. 

\section{Future work}
The heuristic patterns for completely unrolled loops could be extended to 
involve loops whose bound is a known variable which is not changed in the body. 
Furthermore, even more general rules would be beneficial: consider loops where  
the value variables are known at the beginning and they are affected by a known 
constant change by every iteration. These improvements have not been implemented
yet due to some technical and framework limitations.

During the widening process we invalidate any possibly changed information. 
However, a change made on a pointer could mean that we need to 
invalidate all variables due to the lack of advanced pointer analysis. Therefore, 
introducing pointer analysis algorithms to the analyzer could help to develop a 
more precise invalidation process.

The infrastructural improvements enable the analyzer to provide entry points 
for bug finding modules (checkers) on loop entrances/exits and identify the 
currently simulated loop for every \texttt{ExplodedNode}. On top of these entry 
points new checkers can be implemented.

\section{Acknowledgement}
I would like to thank Laszlo Makk and the members of the \texttt{CodeChecker} 
team at Ericsson for their valuable and helpful suggestions on the paper. 

This study was supported by the \'UNKP-17-2 New National Excellence Program of the Hungarian Ministry of Human Capacities.

% The 'abbrvnat' bibliography style is recommended.

%\bibliographystyle{abbrvnat}

\bibliographystyle{apalike}
\bibliography{loopbibliography}

\end{document}
